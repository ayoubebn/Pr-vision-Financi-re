% Options for packages loaded elsewhere
\PassOptionsToPackage{unicode}{hyperref}
\PassOptionsToPackage{hyphens}{url}
%
\documentclass[
]{article}
\usepackage{amsmath,amssymb}
\usepackage{iftex}
\ifPDFTeX
  \usepackage[T1]{fontenc}
  \usepackage[utf8]{inputenc}
  \usepackage{textcomp} % provide euro and other symbols
\else % if luatex or xetex
  \usepackage{unicode-math} % this also loads fontspec
  \defaultfontfeatures{Scale=MatchLowercase}
  \defaultfontfeatures[\rmfamily]{Ligatures=TeX,Scale=1}
\fi
\usepackage{lmodern}
\ifPDFTeX\else
  % xetex/luatex font selection
\fi
% Use upquote if available, for straight quotes in verbatim environments
\IfFileExists{upquote.sty}{\usepackage{upquote}}{}
\IfFileExists{microtype.sty}{% use microtype if available
  \usepackage[]{microtype}
  \UseMicrotypeSet[protrusion]{basicmath} % disable protrusion for tt fonts
}{}
\makeatletter
\@ifundefined{KOMAClassName}{% if non-KOMA class
  \IfFileExists{parskip.sty}{%
    \usepackage{parskip}
  }{% else
    \setlength{\parindent}{0pt}
    \setlength{\parskip}{6pt plus 2pt minus 1pt}}
}{% if KOMA class
  \KOMAoptions{parskip=half}}
\makeatother
\usepackage{xcolor}
\usepackage[margin=1in]{geometry}
\usepackage{graphicx}
\makeatletter
\def\maxwidth{\ifdim\Gin@nat@width>\linewidth\linewidth\else\Gin@nat@width\fi}
\def\maxheight{\ifdim\Gin@nat@height>\textheight\textheight\else\Gin@nat@height\fi}
\makeatother
% Scale images if necessary, so that they will not overflow the page
% margins by default, and it is still possible to overwrite the defaults
% using explicit options in \includegraphics[width, height, ...]{}
\setkeys{Gin}{width=\maxwidth,height=\maxheight,keepaspectratio}
% Set default figure placement to htbp
\makeatletter
\def\fps@figure{htbp}
\makeatother
\setlength{\emergencystretch}{3em} % prevent overfull lines
\providecommand{\tightlist}{%
  \setlength{\itemsep}{0pt}\setlength{\parskip}{0pt}}
\setcounter{secnumdepth}{-\maxdimen} % remove section numbering
\ifLuaTeX
  \usepackage{selnolig}  % disable illegal ligatures
\fi
\usepackage{bookmark}
\IfFileExists{xurl.sty}{\usepackage{xurl}}{} % add URL line breaks if available
\urlstyle{same}
\hypersetup{
  pdftitle={Projet Yahoo Finance},
  hidelinks,
  pdfcreator={LaTeX via pandoc}}

\title{Projet Yahoo Finance}
\author{}
\date{\vspace{-2.5em}}

\begin{document}
\maketitle

\section{Introduction}\label{introduction}

Ce projet analyse les séries temporelles extraites de Yahoo Finance à
l'aide d'une application \textbf{Shiny}. Il comprend les étapes
suivantes : - Importation et visualisation des données. - Analyse des
caractéristiques temporelles (ACF, PACF). - Modélisation des données à
l'aide d'ARIMA et évaluation des performances. - Prédiction des
observations futures.

\section{Objectifs}\label{objectifs}

Les objectifs principaux du projet sont : 1. Extraire et analyser une
série temporelle financière. 2. Appliquer des modèles de séries
temporelles pour prévoir les valeurs futures. 3. Évaluer les modèles en
utilisant les dernières 10\% des observations comme données de test.

\section{Méthodologie}\label{muxe9thodologie}

\subsection{Extraction des données}\label{extraction-des-donnuxe9es}

Les données sont extraites depuis Yahoo Finance via le package
\texttt{quantmod}. Nous utilisons les prix de clôture ajustés
(\texttt{Adjusted\ Closing\ Prices}).

```r library(quantmod)

\section{Paramètres d'exemple}\label{paramuxe8tres-dexemple}

symbol \textless- ``HDFCBANK.NS'' start\_date \textless- ``2023-01-01''
end\_date \textless- ``2024-11-20''

\section{Extraction des données}\label{extraction-des-donnuxe9es-1}

data \textless- getSymbols(symbol, src = ``yahoo'', from = start\_date,
to = end\_date, auto.assign = FALSE) head(data)

\end{document}
